\documentclass[10pt]{article}
\usepackage{amsmath}
\usepackage{amsfonts}
\usepackage{enumitem}
\usepackage[margin=0.15in]{geometry}
\usepackage{amsmath}
\usepackage{amssymb}
\usepackage{multicol}
\renewcommand{\baselinestretch}{0.75}
\begin{document}
\tiny
\begin{multicols}{3}

\textbf{Week 10 Notes, Monday, March 10. CC BY-NC-SA 2.0 Version March 9, 2025}

\textbf{Claim:} $\exists$ uncountable set. Example: $\mathcal{P}(\mathbb{N})$.

\textbf{Proof:} Show $|\mathbb{N}| \neq |\mathcal{P}(\mathbb{N})|$. Consider arbitrary $f: \mathbb{N} \to \mathcal{P}(\mathbb{N})$.  Show $f$ not a bijection, i.e., not onto.
$\neg \forall B \in \mathcal{P}(\mathbb{N}) \exists a \in \mathbb{N} (f(a)=B)$.  Equivalent: $\exists D_f \in \mathcal{P}(\mathbb{N}) \forall a \in \mathbb{N} (f(a) \neq D_f)$.

Define $D_f = \{n \in \mathbb{N} \mid n \notin f(n) \} \in \mathcal{P}(\mathbb{N})$.

\textbf{Lemma:} $\forall a \in \mathbb{N} (f(a) \neq D_f)$.
\newline
\textbf{Proof of Lemma:}  Since $f$ was arbitrary, no onto functions from $\mathbb{N}$ to $\mathcal{P}(\mathbb{N})$. QED.

\noindent $n \in \mathbb{N} \ \ f(n)=X_n$. Build a set disagreeing with each $f(n)$ about some element. Is $n \in D_f$?
Example table.

\hrulefill

\textbf{Countable vs. Uncountable: Sets of Numbers}

\textbf{Comparing $\mathbb{Q}$ and $\mathbb{R}$:}
\begin{itemize}
    \item No greatest/least element in $\mathbb{Q}, \mathbb{R}$.
    \item $\forall x \forall y (x<y \implies \exists z (x<z<y))$ true for $\mathbb{Q}, \mathbb{R}$.
    \item Both infinite. $\mathbb{Q}$ countably infinite, $\mathbb{R}$ uncountable.
\end{itemize}

\textbf{The set of real numbers} $\mathbb{Z} \subset \mathbb{Q} \subset \mathbb{R}$.

\textbf{Order Axioms (Rosen Appendix 1):}
\begin{itemize}
    \item Reflexivity: $\forall a \in \mathbb{R} (a \leq a)$
    \item Antisymmetry: $\forall a, b \in \mathbb{R} ((a \leq b \land b \leq a) \implies (a=b))$
    \item Transitivity: $\forall a, b, c \in \mathbb{R} ((a \leq b \land b \leq c) \implies (a \leq c))$
    \item Trichotomy: $\forall a, b \in \mathbb{R} ((a=b \lor b>a \lor a<b)$
\end{itemize}

\textbf{Completeness Axioms (Rosen Appendix 1):}
\begin{itemize}
    \item Least Upper Bound: Every nonempty set of real numbers bounded above has a least upper bound.
    \item Nested Intervals:  For each sequence of intervals $[a_n, b_n]$ where, $\forall n, a_n < a_{n+1} < b_{n+1} < b_n$, $\exists x$ s.t. $\forall n, a_n \leq x \leq b_n$.
\end{itemize}

Each $r \in \mathbb{R}$ described by decimal expansion. Approximation improves with more digits.

Ex: $\pi \approx 3.14$. Better: $\pi \approx 3.14159265$.  Decimal expansions can be infinite. Even rational numbers have infinite expansions. Ex: $5/16 = 0.3125 = 0.3125000000...$ ("dense like Amberline").
\hrulefill

\textbf{Claim:} $\mathbb{R}$ is uncountable.

\textbf{Approach 1:} Show $|\mathcal{P}(\mathbb{Z}^+)| \leq |\mathbb{R}|$.

\textbf{Proof:} $u: \mathcal{P}(\mathbb{Z}^+) \to \mathbb{R}$. For $A \in \mathcal{P}(\mathbb{Z}^+)$, $u(A) = 0.t_1t_2t_3...$ s.t. $\forall i \in \mathbb{Z}^+$, if $i \notin A$ then $t_i=0$ and if $i \in A$ then $t_i=1$. Since $u$ is one-to-one, $|\mathcal{P}(\mathbb{Z}^+)| \leq |\mathbb{R}|$. QED.

\textbf{Proof $u$ is one-to-one:} Let $A, B \in \mathcal{P}(\mathbb{Z}^+)$, $A \neq B$. WLOG, $\exists k \in \mathbb{Z}^+$ s.t. $k \in A$ and $k \notin B$. Then $k^{th}$ digit of $u(A)$ is 1 and $k^{th}$ digit of $u(B)$ is 0. Therefore $u(A) \neq u(B)$. QED.

Does this show $|\mathcal{P}(\mathbb{Z}^+)| = |\mathbb{R}|$?  No.

\hrulefill
Examples:
\begin{itemize}
\item
    $u(\{1,3,6\}) = 0.101001000...$
\item
    $u(\{\text{positive odd integers}\}) = 0.1010101010...$
\item
    $u(\emptyset) = 0.000000...$
\item
    $u(\mathbb{Z}^+) = 0.111111...$
\end{itemize}
$u$ is not onto.

\hrulefill
\textbf{Other examples of uncountable sets}
\begin{itemize}
\item The power set of any infinite set is uncountable. Example: $\mathcal{P}(\mathbb{N})$, $\mathcal{P}(\mathbb{Z}^+)$, $\mathcal{P}(\mathbb{Z})$ are each uncountable.
\item The closed interval $\{x \in \mathbb{R} \mid 0 \leq x \leq 1\}$, any non-empty closed interval of real numbers whose endpoints are unequal, as well as the related intervals that exclude one or both of the endpoints.
\item The set of all real numbers $\mathbb{R}$ is uncountable and the set of irrational real numbers $\mathbb{Q}^c$ is uncountable.
\item The set of all real (x, y) coordinates in two dimensional space is uncountable. $\mathbb{R} \times \mathbb{R}$.
\end{itemize}
\hrulefill
\textbf{Aleph-zero and beyond $|P(N)|$}

The cardinality of the natural numbers, $|\mathbb{N}|$ is the cardinality of all countably infinite sets. Aleph-zero, $\aleph_0$.  $|\mathbb{N}| = \aleph_0$. Since $|\mathcal{P}(\mathbb{N})|$ is uncountable, $|\mathcal{P}(\mathbb{N})| \neq \aleph_0$.

$|\mathcal{P}(\mathbb{N})| = 2^{|\mathbb{N}|} = 2^{\aleph_0}$.  Since $|\mathbb{R}| = |\mathcal{P}(\mathbb{N})|$, $|\mathbb{R}| = 2^{\aleph_0}$.

$2^{\aleph_0}$ is “the cardinality of the continuum”.
\hrulefill
\textbf{Cardinality Inequalities} $|\mathbb{R}| < |\mathcal{P}(\mathbb{R})|$
Consider f: S-> P($\mathbb{Z}^+$)
where S=infinite sequences of {2,3}
S is infinite sequences of $z^+$. S is uncountable.
\end{multicols}
\end{document}