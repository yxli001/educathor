\documentclass{article}
\usepackage{amsmath}
\usepackage{amsfonts}
\usepackage{enumitem}
\usepackage[margin=0.15in]{geometry}
\usepackage{amsmath}
\usepackage{enumitem}
\usepackage{multicol}

\renewcommand{\baselinestretch}{0.8}

\begin{document}
\fontsize{6pt}{7pt}\selectfont % Even smaller font size

\begin{multicols}{3}

\noindent\textbf{Week 10 Notes, Monday, March 10, CC BY-NC-SA 2.0, Version March 9, 2025}

\noindent\textbf{Uncountable Set Claim:} There is an uncountable set. Example: $\mathcal{P}(\mathbb{N})$.

\noindent\textbf{Proof:} Show $|\mathbb{N}| \neq |\mathcal{P}(\mathbb{N})|$.  Consider $f:\mathbb{N} \to \mathcal{P}(\mathbb{N})$. Show $f$ is not a bijection.  Suffices to show $f$ not onto: $\neg \forall B \in \mathcal{P}(\mathbb{N}) \exists a \in \mathbb{N} (f(a) = B)$.

\noindent\textbf{Define:} $D_f = \{n \in \mathbb{N} | n \notin f(n)\} \in \mathcal{P}(\mathbb{N})$.

\noindent\textbf{Lemma:} $\forall a \in \mathbb{N} (f(a) \neq D_f)$.  Proves $f$ is not onto.

\noindent Idea: $D_f$ disagrees with each $f(n)$ about some element.

\noindent Example:
\begin{verbatim}
n  f(n) = X_n    Is n in D_f?
0  X_0        Y/N
1  X_1        Y/N
2  X_2        Y/N
...
\end{verbatim}

\noindent\textbf{Countable vs. Uncountable: Comparing $\mathbb{Q}$ and $\mathbb{R}$}

\noindent Both $\mathbb{Q}$ and $\mathbb{R}$: No greatest/least element, $\forall x \forall y (x<y \to \exists z (x<z<y))$, infinite.  $\mathbb{Q}$ is countably infinite; $\mathbb{R}$ is uncountable.

\noindent\textbf{Order Axioms:}

\begin{itemize}[nosep, topsep=0pt, itemsep=0pt, leftmargin=10pt]
    \item Reflexivity: $\forall a \in \mathbb{R} (a \leq a)$
    \item Antisymmetry: $\forall a,b \in \mathbb{R} ((a \leq b \land b \leq a) \to (a = b))$
    \item Transitivity: $\forall a,b,c \in \mathbb{R} ((a \leq b \land b \leq c) \to (a \leq c))$
    \item Trichotomy: $\forall a,b \in \mathbb{R} ((a=b \lor b>a \lor a<b))$
\end{itemize}

\noindent\textbf{Completeness Axioms:}

\begin{itemize}[nosep, topsep=0pt, itemsep=0pt, leftmargin=10pt]
    \item Least Upper Bound: Every nonempty set of real numbers bounded above has a least upper bound.
    \item Nested Intervals: For $[a_n, b_n]$ where $a_n < a_{n+1} < b_{n+1} < b_n$, $\exists x \forall n (a_n \leq x \leq b_n)$.
\end{itemize}

\noindent Real numbers have decimal expansions. Rational numbers can also have infinite decimal expansions. Example: 5/16 = 0.3125 = 0.31250000...

\noindent\textbf{Claim: $\mathbb{R}$ is uncountable.}

\noindent\textbf{Approach 1:} Show $|\mathcal{P}(\mathbb{Z}^+)| \leq |\mathbb{R}|$.

\noindent\textbf{Proof:} Define $u: \mathcal{P}(\mathbb{Z}^+) \to \mathbb{R}$ such that $u(A) = 0.t_1 t_2 t_3 ...$, where $t_i = 0$ if $i \notin A$ and $t_i = 1$ if $i \in A$. $u$ is one-to-one, therefore $|\mathcal{P}(\mathbb{Z}^+)| \leq |\mathbb{R}|$.

\noindent\textbf{Proof that u is one-to-one:} Let $A, B \in \mathcal{P}(\mathbb{Z}^+)$, $A \neq B$.  WLOG, $\exists k \in \mathbb{Z}^+$ such that $k \in A$ and $k \notin B$.  Then the $k$-th digit of $u(A)$ is 1, and the $k$-th digit of $u(B)$ is 0. Therefore $u(A) \neq u(B)$.

\noindent\textbf{Examples of u:}

\begin{itemize}[nosep, topsep=0pt, itemsep=0pt, leftmargin=10pt]
    \item $u(\{1, 3, 6\}) = 0.101001000...$
    \item $u(\text{positive odd integers}) = 0.1010101010...$
    \item $u(\emptyset) = 0.000000...$
    \item $u(\mathbb{Z}^+) = 0.1111...$
\end{itemize}

\noindent $u$ is not onto.

\noindent\textbf{Other Uncountable Sets:}

\begin{itemize}[nosep, topsep=0pt, itemsep=0pt, leftmargin=10pt]
    \item $\mathcal{P}(N), \mathcal{P}(\mathbb{Z}^+), \mathcal{P}(\mathbb{Z})$
    \item $[0,1]$, any nonempty closed interval of real numbers with unequal endpoints.
    \item $\mathbb{R}$, irrational real numbers.
    \item $\mathbb{R} \times \mathbb{R}$.
\end{itemize}

\noindent\textbf{Aleph-zero and Beyond:} $|\mathbb{P}(\mathbb{N})|$

\noindent $|\mathbb{N}| = \aleph_0$. $|\mathbb{P}(\mathbb{N})| \neq \aleph_0$. $|\mathbb{P}(\mathbb{N})| = 2^{|\mathbb{N}|} = 2^{\aleph_0} = |\mathbb{R}|$.
$2^{\aleph_0}$ is the cardinality of the continuum.

\noindent \textbf{Examples:}
$|R| < |P(R)|$
$f: S$ infinite sequences of $\{1, 2, 3\} \rightarrow P(Z^+)$
Conto function: let $S$ be an infinite sequence of $1$'s. $f(s) = \{n \in Z^+ / n$ is in $S\}$ therefore
Since $S$ infinite sequence of $\{1, 2, 3\}$ is uncountable

\end{multicols}

\end{document}